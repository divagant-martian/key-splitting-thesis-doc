\chapter{Preliminaries
  \label{chapter:preliminaries}}

\comment{
  Put down everything I think should be here. Order it, make subsections for each. Start filling the gaps.
  Pags: 3- 17 min
}

\section{Paradigm Shift}
What is a stream in the world
\section{Stream Modeling}
DAGS, unbounded tables
\section{General architecture of a DSPS}
\subsection{Logical Plan}
\subsection{Physical Plan}
\section{Load Balancing Optimizations}
\subsection{Substream spliting}
\subsubsection{Key spliting}
\subsubsubsection{Power of W choices}
\subsubsubsection{Hashing}
\subsubsubsection{The others}
\subsubsection{Windw spliting (and others?)}
\section{Other optimizations}


\section{Paradigm Shift}
Paradigm Shift\cite{chakravarthy2009stream}\\ As a result, the amount of data to be
processed can be unbounded or never ending. At the same time, these applications
need processing capabilities for continuously computing and aggregating incoming
data for identifying interesting changes or patterns in a timely manner.

These applications are different from traditional DBMS applications with respect
to data arrival rates, update frequency, processing requirements, Quality of
Service (QoS) needs, and notification support

Queries that are processed by a traditional DBMS are termed ad hoc queries. They
are (typically) specified, optimized, and evaluated once over a snapshot of a
database. In contrast, queries in a stream processing environment are termed
continuous queries

In addition, the snapshot approach for evaluating stream data may not always be
appropriate as the values over an interval are important (e.g., temperature
changes) for stream processing applications. Furthermore, the inability to
specify quality of service requirements (such as latency or response time) to a
DBMS makes its usage less acceptable for stream applications.

DBMSs were not designed to manage high-frequency updates (in the form of data
streams) and to provide continuous computation and output for queries.

Applications\cite{chakravarthy2009stream}\\ i) to monitor traffic slowdown or
accidents using data sent by each car on the road every few seconds or minutes,
(ii) to perform program trading based on changes in the stock price of a
particular stock relative to other stock prices using data from multiple feeds,
and (iii) to monitor environmental and security applications for water
quality, fire spread, etc. based on data received from sensors

Data Stream Characteristics\cite{chakravarthy2009stream}\\


Data Stream Application Characteristics\cite{chakravarthy2009stream}\\


Continuous Queries\cite{chakravarthy2009stream}\\


Window Specification\cite{chakravarthy2009stream}\\


QoS Metrics\cite{chakravarthy2009stream}\\


Data Stream Management System Architecture\cite{chakravarthy2009stream}\\


QoS-Related Challenges\cite{chakravarthy2009stream}\\


Capacity Planning and QoS Verification\cite{chakravarthy2009stream}\\


Scheduling Strategies for CQs\cite{chakravarthy2009stream}\\


Load Shedding and Run-Time Optimization\cite{chakravarthy2009stream}\\


Complex Event and Rule Processing\cite{chakravarthy2009stream}\\


Design and Implementation of a DSMS with CEP\cite{chakravarthy2009stream}\\


Stream Processing Model\cite{kamburugamuve2013survey}\\


Physical Distribution of PEs\cite{kamburugamuve2013survey}\\


Stream Processing Engine Requirements\cite{kamburugamuve2013survey}\\


Fault Tolerance\cite{kamburugamuve2013survey}\\


GAP Recovery\cite{kamburugamuve2013survey}\\


Rollback Recovery\cite{kamburugamuve2013survey}\\


Upstream Backup\cite{kamburugamuve2013survey}\\


Precise Recovery\cite{kamburugamuve2013survey}\\



Operator Graphs\cite{schneider2013tutorial}\\


State in Operators\cite{schneider2013tutorial}\\


Flavors of Parallelism\cite{schneider2013tutorial}\\


Safety and Profitability\cite{schneider2013tutorial}\\
