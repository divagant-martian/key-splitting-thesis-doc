% !TEX root = ../main.tex
\chapter{Preliminaries
  \label{chapter:preliminaries}}

  This section is a primer on stream processing. It covers the formalization of streams and related concepts, as well as the basics of the main characteristics that shape the general architecture of a Data Stream Processing System (DSPS). Subsequently, it provides an overview on the most common Quality of Service metrics, and how these are affected by architectural decisions, particularly regarding elasticity and distribution. It concludes with the definition of load balancing and the techniques used to handle this challenge, with a great emphasis on key splitting.

  \section{Data Stream Modeling}

  % Streams
  Decir qué son streams. Decir que se modelan como tuplas de tamaño variable. \cite{kamburugamuve2013survey} Decir que en algunos casos se supone cierta estructura y se considern tablas no acotadas.\cite{KottoKombi2015ParallelAD}

  % Operator
  Decir que estos streams son procesados por unidades funcionales llamadas operadores. Decir que estas unidades son lógicas y section les conoce también como Elementos de procesamiento \cite{kamburugamuve2013survey} Dar ejemplos de qué es un operador, y decir que se hace una distinción importante entre si un operador maneja o no maneja estado. Mostrar ejemplos de un operador con estado y uno sin estado.
  \cite{KottoKombi2015ParallelAD}

  % General system model
  La definición lógica del flujo de streams a través de los operadores se modela como un DAG. Este DAG incluye las fuentes y los sinks. Dar un ejemplo dibujado de un DAG y decir que el sistema puede emitir o no emitir tuplas en respuesta a un stram entrante.
  % DAG
  % DAG  \cite{kamburugamuve2013survey}
  % Operator Graphs \cite{R_ger_2019}
  % Workflow  \cite{KottoKombi2015ParallelAD}

  Explicar primero qué es distribución. Explicar que este DAG solo representa de forma lógica el flujo de datos. Hablar del plan físico y explicar cómo se mapea el plan lógico al físico. Explicar qué es paralelismo y cómo se ve el plan físico a la luz del paralelismo. Incluir un dibujito.
  \cite{kamburugamuve2013survey}

  Decir que el paralelismo, en conjunto con la distribución ponen sobre la mesa características interesante de los sistemas de procesamientos de datos, o hacen que determinadas decisiones del modelo afecten sus capacidades de paralelismo. Explicar cuáles son estas caracterísicas: Type of SP System, Programming Model, Sub-Stream Processing, Modeo de insfraestructura, Manejo de estado en operadores. \cite{R_ger_2019} Decir por qué en cada clasificación se escogió el modelo que se escogió para trabajar bajo las garantías del sistema.

  Explicar que el paralelismo se puede evidenciar en dos modos a grandes rasgos: paralelismo de tareas o de datos. Explicar brevemente paralelismo de tareas y entrar en detalle en paralelismo de datos
  Explicar qué es elasticidad

  Parallelization and Elasticity methods \cite{R_ger_2019}\\
  - Task parallelization \cite{R_ger_2019}\\
  - Shuffle grouping \cite{R_ger_2019}\\
  - Key partitioning functions \cite{R_ger_2019}\\
  - Key-based splitting: State Management and Algorithms \cite{R_ger_2019}\\
  - Operator Elasticity methods \cite{R_ger_2019}
  - Operator parallelization methods \cite{R_ger_2019}

  % Parallel Data Stream Processing Engines classification \cite{R_ger_2019}

  \section{QoS Metrics}
  QoS-Related Challenges\cite{chakravarthy2009stream}
  \section{Load Balancing in the context of elasticity and distribution}

  Problem description: Load Balancing  \cite{Hirzel_2014},   Load Balancing \cite{R_ger_2019}

  State management (considerations): State in Operators \cite{R_ger_2019}. State management \cite{R_ger_2019}: Safety and Profitability \cite{Schneider_2013}\cite{R_ger_2019}

  Other optimizations

  Comparison between most common engines taking into account all that was said in this section before
  Most common Engines  \cite{kamburugamuve2013survey}
  - Storm  \cite{kamburugamuve2013survey} \cite{R_ger_2019}\\
  - Heron \cite{R_ger_2019}\\
  - Spark \cite{R_ger_2019}\\
  - Flink \cite{R_ger_2019}

  Conclusion: Open Research Problems \cite{Schneider_2013}

